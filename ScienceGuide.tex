\documentclass[9pt]{article}
\usepackage[utf8]{inputenc}

\title{Science Semester II Study Guide}
\author{Nikhil Gaitonde }
\date{May 2019}
\addtolength{\oddsidemargin}{-1.5in}
\addtolength{\evensidemargin}{-1.5in}
\addtolength{\textwidth}{2.5in}

\addtolength{\topmargin}{-.9in}
\addtolength{\textheight}{2.5in}
\begin{document}

\maketitle

\section*{Chapter 11}
\begin{enumerate}
\item {\bf Gene}
A unit of heredity which is transferred from a parent to offspring. A distinct sequence of nucleotides forming part of a chromosome, the order of which determines the order of monomers in a polypeptide which a cell may synthesize.
\item {\bf Allele}
One of two or more alternative forms of a gene that arise by mutation and are found at the same place on a chromosome.
\item {\bf Hybrid}
The offspring of crosses between parents with different traits
\item {\bf Genome}
The complete set of genes or genetic material present in a cell or organism
\item {\bf Dominant allele}
One allele of a gene is expressed even if another allele is present. The dominant allele masks the expression of the other allele.
\item {\bf Recessive allele}
This form of allele will be expressed only if the other pair is the same type. If the other pair is dominant, then the recessive will not be expressed.
\item {\bf Heterozygous}
Organisms that have two {\em different} alleles for the same trait.
\item {\bf Homozygous}
Organisms that have two {\em identical} alleles for a particular trait.
\item {\bf Principle of Dominance} If two or more forms (alleles) of
  the gene for a single trait exist, some forms of the gene may be
  dominant and others may be recessive.
\item {\bf Inheritance pattern of dominant allele} - always shows up in allele
\item {\bf Inheritance pattern of recessive allele} - only shows up if both recessive alleles are present
\item {\bf Law of Segregation} Each adult has two copies of each
  gene. These genes are segregated from each other when gametes are
  formed.
\item {\bf Law of Independent Assortment} The alleles for different
  genes usually segregate independently of one another
\item {\bf Genotype} The genetic make-up of an organism  -  shows the
  combination of alleles (ex: PP or Pp or pp)
\item {\bf Phenotype} The characteristics or traits that show up - the
  appearance of an organism(ex: purple flower or white flower)
\item {\bf Incomplete dominance} When one allele is NOT completely
  dominant over another (they blend - pink carnations)
\item {\bf Codominance} Both alleles are expressed in the phenotype
  (black and white chicken)
\item {\bf Multiple alleles} Genes that are controlled by more than
  two alleles
\item {\bf Alleles for the blood groups (genotypes and phenotypes)}
  Three alleles for this gene
  \begin{itemize}
    \item $I^A$, $I^B$ and $i$
    \item Alleles $I^A$ and $I^B$ are codominant
    \item $i$ is recessive
  \end{itemize}
\item {\bf Diploid number vs Haploid number} Haploid is Total number
  of chromosomes found in a gamete (23 for humans). Humans are diploid
  organisms, carrying two complete sets of chromosomes: one set of 23
  chromosomes from their father and one set of 23 chromosomes from
  their mother.
\item {\bf Somatic cell vs sex cell} - somatic cell is is the body cell while sex cell is either x or y
\item {\bf Importance of meiosis} - creates diversity
\item {\bf Homologous chromosomes} Homologous chromosomes are
  chromosome pairs (one from each parent) that are similar in length,
  gene position, and location. The position of the genes on each
  homologous chromosome is the same. However, the genes may contain
  different alleles.
\item {\bf Tetrad}
  \begin{itemize}
    \item Each chromosome pairs with its corresponding homologous
      chromosome to form a tetrad. This process is called Synapsis.
    \item There are 4 chromatids in a tetrad.
  \end{itemize}
\item {\bf Importance of crossing over}
  \begin{itemize}
    \item Homologous chromosomes form a tetrad
    \item Chromatids cross over one another
    \item The crossed sections of the chromatids are exchanged.
    \item Crossing-over produces new combinations of alleles.
  \end{itemize}
  {\bf \it Understand from pictures}
\item {\bf Gamete formation in males} In male animals, meiosis results
  in four equalsized gametes called sperm. {\bf \it Understand from
    pictures}
\item {\bf Gamete formation in females} Only one egg results from
  meiosis. The other three cells, called polar bodies, are usually not
  involved in reproduction.  They give up their cytoplasm to nourish
  the 1 good egg. {\bf \it Understand from pictures}
\item {\bf Should be able to}
  \begin{itemize}
    \item {\em set up and interpret a monohybrid cross and a dihybrid cross}
    \item {\em predict probability of offspring from Punnett square}
  \end{itemize}
\end{enumerate}
\section*{Chapter 12}
\begin{enumerate}
  \item {\bf Structure of DNA (a drawing will help you remember)} - deoxyribose,phosphate group,nitrogenous base
  \item {\bf Parts of a nucleotide} Monomer of nucleic acids made up of:
    \begin{itemize}
      \item 5-carbon Sugar
      \item Phosphate Group
      \item Nitrogenous Base
    \end{itemize}
  \item {\bf Base pairing rule} - Adenine pairs with Thymine and Cytosine pairs with Guanine
  \item {\bf Purines} Double ring bases. Adenine, Guanine.
  \item {\bf Pyrimidines} Single ring bases. Cytosine, Thymine.
  \item {\bf Structure of RNA} - Ribose,phosphate group,nitrogenous base
  \item {\bf Difference between RNA and DNA}
  RNA
  \begin{itemize}
    \item Number of strands = One strand
    \item Type of sugar = Ribose
    \item Nitorgen bases =  AUCG
    \item Location = Nucleus, cytoplasm, ribosome
  \end{itemize}
  DNA
  \begin{itemize}
    \item Number of strands = 2 strands
    \item Type of sugar = Deoxyribose
    \item Nitrogen bases = AUCG
    \item Location = Nucleus
  \end{itemize}
  \item {\bf Types of RNA and functions  -  mRNA, rRNA, tRNA}
  
  mRNA
  \begin{itemize}
    \item Messenger RNA
    \item Carries DNA copies to rest of the cell
  \end{itemize}
  rRNA
  \begin{itemize}
    \item Ribosomal RNA
    \item what a ribosome is made up of
  \end{itemize}
  tRNA
  \begin{itemize}
    \item Trannsfer RNA
    \item transfers each amino acid to the ribosome according to the message received from mRNA
  \end{itemize}
  \item {\bf Codon} - 3 consecutive nucletides on the mRNA that stand for one amino acid
  \item {\bf Stop codon} Stop transcription here. UAA, UAG, UGA
  \item {\bf Start codons} Start transcription here. AUG or methionine.
  \item {\bf Anticodon} - 3 bases that are complementary to one mRNA codon
  \item {\bf DNA replication (where it takes place, what it is, enzymes involved)}
  
  Prokaryote
  \begin{itemize}
    \item starts at one point and continues until entire chromosome is replicated
  \end{itemize}
  Eukaryote
  \begin{itemize}
    \item occurs at hundred of places
    \item proceeds in both directions until each chromosome is completely copied
    \item sites where rpelication and seperation occur are called replication forks
  \end{itemize}
  what it is - duplication of DNA
  enzymes involved
  \begin{itemize}
    \item Helicase
    \item DNA polymerase
    \item DNA ligase
  \end{itemize}
  \item {\bf Transcription (where it takes place, what it is, structures and enzymes involved)}
  where it takes place - takes place in nucleus
  what it is - DNA is copied in form of RNA
  Structures and enzymes involved - RNA polymerase
  How it works
  \begin{itemize}
    \item begins at a promoter
    \item RNA polymerase binds to a promoter
    \item  This signals the DNA to unwind so the enzyme can‘‘read’’ the bases in one of the DNA strand
    \item The enzyme is now ready to make a strand ofmRNA with a complementary sequence of bases
    \item Transcription ends when RNA polymerase crosses a stop (termination) sequence in the gene. The mRNA strand is complete, and it detaches from DNA.
    \item  The DNA that has already been “read” zips back up into a double helix structure
  \end{itemize}
  \item {\bf Translation (what this is, structures involved, where it takes place)}
  where it takes place - cytoplasm
  what it is  - synthesis of proteins
  structures involved - mRNA,tRNA
  \item {\bf Should be able to}
    \begin{itemize}
      \item {\em make a complementary DNA strand when given bases of one strand}
      \item {\em be able to figure out the DNA, mRNA, tRNA sequences and polypeptide chain when either is given}
      \item {\em read the Genetic Code}
      \item {\em Recognize a mutation as substitution, nonsense, silent, frameshift, insertion, deletion}
    \end{itemize}
\end{enumerate}
\section*{Chapter 13}
\begin{enumerate}
  \item {\bf Selective breeding} Breeding only those plants or animals
    with desirable traits.
  \item {\bf Hybridization} Crossing of dissimilar individuals to
    bring together the best of both organisms.
  \item {\bf Inbreeding} Continued breeding of individuals with
    similar characteristics, organisms are genetically similar.
  \item {\bf Restriction enzymes (Where are they found? How they work?
    Blunt End vs Sticky End)} Prokaryotic enzymes that recognize and
    cut DNA at specific sequences, called restriction sites. Work by
    chopping up the foreign DNA. Sticky Ends: Single ended strands
    left by restriction enzymes for some other strand to bind to
    it. Blunt Ends: Some Restriction Enzymes leave no single stranded,
    but regular double sttanded. Not sticky.
  \item {\bf Gel electrophoresis  -  purpose and process}

  purpose - to sepereate DNA strands based on their size
  
  process
  \begin{emumerate}
    \item Restriction enzymes cut DNA into fragments
    \item fragments are poured into wells on a gel
    \item voltage is applied to gel
    \item DNA is negatively charged so move toward positive ends of gel
    \item the smaller the fragment, the faster and further it will move across the gel
  \end{enumerate}
  \item {\bf DNA Fingerprinting and how it works} - analyzes DNA sections that have little or no known function but vary widely from one individual to another
  
  process
  \begin{enumerate}
    \item Restriction enzymes are used to cut the DNA into fragmetns containing genes and repeats
    \item fragments containing repeats are labeled
  \end{enumerate}
  \item {\bf Plasmid} Small circular bacterial DNA.
  \item {\bf Recombinant DNA  -  process; examples; enzymes involved}

  process
  \begin{enumerate}
    \item remove bacterial plasmid
    \item cut plasmid with restriction enzymes
    \item cut DNA from another organism with restriction enzymes
    \item Combine the cut pieces of DNA together with the enzyme ligase and insert the recombinant DNA into bacteria
    \item reproduce recombinant bacteria
    \item foreign genes will be expressed
  \end{enumerate}
  examples - Growth hormone, insulin
  enzymes involved - DNA ligase, restriction enzymes, DNA plasmid
  \item {\bf Transgenic organisms} - organism that contians genes from another species
  \item {\bf Cloning (what this is, not the process)} - producing an exact genetic copy of an organsim
\end{enumerate}
\section*{Chapter 14}
\begin{enumerate}
  \item {\bf Karyotype and its purpose}  Cells are photographed during mitosis, a picture of chromosomes grouped in pairs able to show changes in chromosomes
  \item {\bf Autosomes vs Sex chromosomes}
  Autosomes - 44 autosomal chromosomes
  Sex Chromosomes - 2 chromosomes X and Y 
  Male: XY
  Female: XX
  \item {\bf Biological sex determination in humans} child sex determined by father 
  \item {\bf Rh factor in human blood}
  determined by prescence or abscence of Rh protein on red blood cell surface
  \item {\bf Blood type inheritance - which alleles are involved; which are dominant and recessive; antigens and antibodies}
  \begin{itemize}
  \item 4 groups - A, B, AB and O
  \item AB group - three alleles $I^A$, $I^B$, and $i$
  \item $I^A$ and $I^B$ are codominant, while allele $i$ is recessive
  \item antigen is A and B
  \item A group - allele $I^A$ with $I^A$ or $I^A$ with $i$ produce antigen A
  \item B group - allele $I^B$ with $I^B$ or $I^B$ with $i$ antigen B
  \item O group - allele $i$ with $i$ produce no antigen
  \end{itemize}
  \item {\bf Why is X-linked recessive inheritance more common in males?}
  because they have only one X chromosome
  \item {\bf Non disjunction}
  the failure of one or more pairs of homologous chromosomes or sister chromatids to separate normally during nuclear division, usually resulting in an abnormal distribution of chromosomes in the daughter nuclei.
  \item {\bf Should be able to}
    \begin{itemize}
      \item {\em Interpret and create pedigree chart  -  autosomal dominant, recessive, X-linked recessive}
      \item {\em Interpret a karyotype}
    \end{itemize}
\end{enumerate}

\section*{Chapter 15}
\begin{enumerate}
  \item {\bf Lamarck's hypothesis of evolution} All organisms have a
    tendency to be perfect. They are continuously changing and
    acquiring new features to live successfully in their
    environments. Example: Bird ancestors desired to fly so they tried
    until wings developed
  \item {\bf Struggle for existence?} Members of each species compete
    regularly for food, living space, and other life necessities.
  \item {\bf Competition} Organisims competing for limited resources or for mating
  \item {\bf The definition of fitness according to Darwin} Ability of an individual to survive and reproduce
  \item {\bf Survival of the fittest} Those best suited for
    environment, survive and reproduce.
  \item {\bf Four main steps of Natural selection with their definition}
    \begin{enumerate}
    \item Overproduction: Have more kids than can survive
    \item Variation: Individuals differ
    \item Selection: Different survival probability
    \item Traits of surviving individuals become more common.
    \end{enumerate}
  \item {\bf The definition of and types of sexual selection} Two types

  Intersexual : Mate Choice (females choosing a mate)

  Intrasexual : Competition for mates (males fighting among themselves for a female)
  \item {\bf Descent with modification} Each living species has descended by changes from another species over time.
  \item {\bf Homologous structures} Bodily structures similar
    in structure due to sharing a common ancestor, but different in
    function. (e.g hands of mammals). Evidence of common ancestory.
  \item {\bf Analogous structures} Bodily structures similar in
    function, but not in structure.  {\em NOT EVIDENCE OF COMMON ANCESTRY
    but of evolution.}
  \item {\bf Vestigial organs} Structures that are present but
    diminished in size or function. Give evolutionary history. Long
    time ago, must have been useful.
  \item {\bf The five main points of Darwin's theory}
    \begin{enumerate}
    \item Individual differ, and some of this variation is heritable
    \item Offspring produce more offspring than can survive; those
      that don’t survive do not reproduce
    \item Due to limited resources, organisms compete
    \item Natural Selection: Individuals best suited to their
      environment survive and reproduce thus passing their heritable
      traits to the offspring. Others die or leave fewer
      offspring. Species change over time.
    \item Common ancestory for all species.
    \end{enumerate}
  \item {\bf DNA evidence of evolution} Amino acid sequences for a protein is compared between organisms. Similar sequence means closely related.
\end{enumerate}
\section*{Chapter 16: 16-3 The Process of Speciation}
\begin{enumerate}
  \item {\bf Reproductive isolation} When the members of two
    populations cannot interbreed and produce fertile offspring.
  \item {\bf Behavioral isolation} Two populations can interbreed but
    have different mating rituals or strategies.
  \item {\bf Geographic isolation} Two populations separated by
    geographic barriers such as rivers or mountains.
  \item {\bf Temporal isolation} Two or more species with overlapping
    range, reproduce at different times. Mating seasons are different.
\end{enumerate}
\section*{Chapter 17}
\begin{enumerate}
  \item {\bf Fossils and how they form} Preserved remains of organisms from remote past. Mostly in sedimentary rocks. 

  Organism dies. Body combines with sedimentary rocks. Remains buried together with sediments. 
  Water minerals enter fossil and harden the fossil. Layers of sediment keep adding up. Weight and pressure converts fossil to rock.

  \item {\bf Relative dating} Age of a fossil is determined by
    comparing its placement with that of fossils in other layers of
    rock.
  \item {\bf Index fossil} Used to define and identify geologic
    periods. Must have a short vertical range, wide geographic
    distribution and rapid evolutionary trends.
  \item  {\bf Half  life}  How long  does it  take  for a  radioactive
    substance to reduce to half its original value.
  \item {\bf Earth's early atmosphere} Hydrogen Cyanide, $CO_2$, CO, N,
    $H_2S$, $H_2O$ vapor but no $O_2$.
  \item {\bf Why life didn't form as soon as the Earth was formed} No $O_2$. Poisonous earth's atmosphere.
  \item {\bf When and under what conditions life started on Earth} 2M to 3M years after earth had liquid $H_2O$. Photosynthetic organisms were to start life.
  \item {\bf Endosymbiotic theory} Eukaryotic cells arose from
    prokaryotic organisms.
    \begin{itemize}
    \item Cell membranes start in prokaryotes. Primitive eukaryotic cell.
    \item Some prokaryotes enter eukaryotes. Symbiosis (joint help) each other.
    \item Inner prokaryotes evolve into organelles.
    \end{itemize}
  \item {\bf Photosynthetic organisms} Use sunlight for energy.
  \item {\bf Chemosynthetic organisms} Use chemicals for energy
  \item {\bf The two major mass extinctions} End of permian and end of
    cretaceous period.
  \item {\bf Adaptive radiation/divergent evolution} Adaptive is a
    type of divergent evolution. Single species evolves into many
    new species to fill available niches (Finches). Divergent:
  \item {\bf Convergent evolution} Structures similar because they
    evolved to do the same job, not because they were inherited from a
    common ancestor.
  \item {\bf Coevolution} Process by which two species evolve in
    response to changes in each other over time.
  \item {\bf Punctuated equilibrium} Long stable periods are
    interrupted by brief periods of more rapid change.
  \item {\bf Should be able to}
    \begin{itemize}
      \item {\em Determine which organism is more evolved when looking at a cladogram}
      \item {\em Determine common ancestors from a cladogram}
      \item {\em Determine what traits organisms share when looking at a cladogram}
      \item {\em Calculate half life}
    \end{itemize}
\end{enumerate}
\section*{Plate Tectonics}
\begin{enumerate}
  \item {\bf Continental crust}
    \begin{itemize}
      \item Thicker
      \item Mostly Granite (igneous rock)
      \item Less dense than oceanic crust
    \end{itemize}
  \item {\bf Oceanic crust}
    \begin{itemize}
      \item Thiner
      \item Mostly Basalt (igneous rock)
      \item More  dense than continental crust
    \end{itemize}
  \item {\bf Convection currents}
    \begin{itemize}
      \item Form of heat transfer. Currents within fluids
      \item Hot. Molecules move apart. Fluid less dense. Rises.
      \item Cold. Molecules move together. Fluid more dense. Sinks
      \item Movement is vertical.
    \end{itemize}
  \item {\bf Continental drift theory} Wegener's hypothesis was that
    all the continents were once joined together in a single landmass
    - Pangaea and have since drifted apart.
  \item {\bf Age of Earth} 
\end{enumerate}
\section*{Chapter 19}
\begin{enumerate}
  \item {\bf Shapes of bacteria}
    \begin{itemize}
    \item Bacili: Rod shaped
    \item Cocci: Spherical
    \item Spirilli: Spiral or Corkscrew shaped prokaryotes
    \end{itemize}
  \item {\bf Gram positive bacteria} Thick cell walls with lots of
    peptidoglycan.  Stain dark violet stain
  \item {\bf Gram negative bacteria} Thinner cell walls. No
    peptidoglycan. Have an outer lipid layer (makes it harder to
    kill). Stain pink
  \item {\bf Structure of virus} Nucleic Acid, Protein and sometimes
    lipid cover.
\end{enumerate}
\section*{Chapter 35-1 Homeostasis}
\begin{enumerate}
  \item {\bf Homeostasis} Process by which organisms keep internal
    conditions mostly constant even if external environment changes.
  \item {\bf Negative feedback loop} Something changes from set
    value. Body makes response to do opposite (negative thing). Makes
    variable get back to set value. ({\em Give examples})
  \item {\bf Levels of organization in human body}
    \begin{itemize}
    \item Cells
    \item Tissues
    \item Organs
    \item Organ Systems      
    \end{itemize}
\end{enumerate}
\section*{Chapter 37}
\begin{enumerate}
  \item {\bf Structure of heart}
  \begin{itemize}
  \item Cardiac Muscle Tissue (Myocardium) forms thick middle layer between outer and inner heart layers
  \item Layers : Outer (Epicardium), Inner (Endocardium)
  \item Pericardium : Membrane enclosing the heart
  \item Septum : Divides left from right
  \item Atria : Upper chambers of the heart that receive blood
  \item Ventricles ; Pump blood out of the heart
  \end{itemize}
  \item {\bf Importance of valves} Prevents blood from flowing back into atria
  \item {\bf Circulation of blood through the heart} 
  \begin{enumerate}
  \item The right atrium receives oxygen-poor blood from the body and pumps it to the right ventricle through the tricuspid valve.
  \item The right ventricle pumps the oxygen-poor blood to the lungs through the pulmonary valve.
  \item The left atrium receives oxygen-rich blood from the lungs and pumps it to the left ventricle through the mitral valve.
  \item The left ventricle pumps the oxygen-rich blood through the aortic valve out to the rest of the body.
  \end{enumerate}
  \item {\bf Pulmonary circulation}

   Circulates blood between the heart and the lungs. 

   Deoxygenated blood leaves the right ventricle through pulmonary arteries, which transport it to the lungs.

   In the lungs, the blood gives up carbon dioxide and picks up oxygen.

   The oxygenated blood then returns to the left atrium of the heart through pulmonary veins
  \item {\bf Systemic circulation} Circulates blood between the heart and the rest of the body. 

  Oxygenated blood leaves the left ventricle through the aorta. The aorta and other arteries transport the blood throughout the body, where it gives up oxygen and picks up carbon dioxide.

  The deoxygenated blood then returns to the right atrium through veins
  \item {\bf Pacemaker} Sino Atrial Node
  \begin{enumerate}
    \item Contraction begins in sino atrial node in right atrium
    \item Impulse spreads from pacemaker to network of fibers in atria causes atria to contract.
    \item Impulse also reaches atrial-ventricular node. Reaches fibers in verntricles, causing them to contract
    \item When atria contracts, blood flows into ventricles
    \item when ventricles contract, blood flows out of heart
  \end{enumerate}
  \item {\bf Arteries} Large vessels that carry blood from the heart to the tissues of the body.

   Except for the pulmonary arteries, all arteries carry oxygen-rich blood.

   Arteries have thick, elastic walls that allow them to withstand the pressure of the blood as the heart contracts
  \item {\bf Veins}  Large veins contain valves that keep blood moving toward the heart.

   Many veins are located near and between skeletal muscles.

   Contraction of skeletal muscles helps move blood in veins toward the heart
  \item {\bf Capillaries} Very narrow (one cell thick). Bring oxygen
    and nutrients to tissues and take waste and carbon dioxide from
    them.
  \item {\bf Systolic pressure} Pressure on the arteries when the
    ventricles contract. Time of highest pressure in the arteries
  \item {\bf Diastolic pressure} When ventricles relax. Lowest
    pressure.
  \item {\bf Red blood cells structure and function} 
  Their flexible disc shape helps to increase the surface area-to-volume ratio
  of these extremely small cells. This enables oxygen and carbon dioxide to
  diffuse across the red blood cell's plasma membrane more readily. Red blood
  cells contain enormous amounts of a protein called hemoglobin. This
  iron-containing molecule binds oxygen as oxygen molecules enter blood vessels
  in the lungs. Hemoglobin is also responsible for the characteristic red color
  of blood. Unlike other cells of the body, mature red blood cells do not
  contain a nucleus, mitochondria, or ribosomes. The absence of these cell
  structures leaves room for the hundreds of millions of hemoglobin molecules
  found in red blood cells. A mutation in the hemoglobin gene can result in the
  development of sickle-shaped cells and lead to sickle cell disorder.
  \item {\bf Blood clotting steps}
  \begin{enumerate}
    \item Blood vessels injured
    \item platelets clump at site form thromboplastin - helps convert prothrombin to thrombin
    \item Thrombin converts fibrinogen into fibers casuing clot
  \end{enumerate}
  \item {\bf Structure of respiratory system} Nose, mouth, epiglottis, pharynx, larynx, trachea, lungs, bronchus, bronchiole and diaphragm
  \begin{itemize}
  \item Upper respiratory tract: Includes the nose, mouth, and the beginning of the trachea (the section that takes air in and lets it out).
  \item Lower respiratory tract: Includes the trachea, the bronchi, broncheoli and the lungs (the act of breathing takes place in this part of the system).
  \item The trachea – the tube connecting the throat to the bronchi.
  \item The bronchi – the trachea divides into two bronchi (tubes). One leads to the left lung, the other to the right lung. Inside the lungs each of the bronchi divides into smaller bronchi.
  \item The broncheoli - the bronchi branches off into smaller tubes called broncheoli which end in the pulmonary alveolus.
  \item Pulmonary alveoli – tiny sacs (air sacs) delineated by a single-layer membrane with blood capillaries at the other end.


  \item The inner surface of the lungs where the exchange of gases takes place is very large, due to the structure of the air sacs of the alveoli.

  \item The lungs – a pair of organs found in all vertebrates. 
The structure of the lungs includes the bronchial tree – air tubes branching off from the bronchi into smaller and smaller air tubes, each one ending in a pulmonary alveolus.
  \end{itemize}

  \item {\bf Path of air flow from nasal cavity to alveoli}
  \item {\bf Gas exchange between alveoli and capillaries} The exchange of gases takes place through the membrane of the pulmonary alveolus, which always contains air: oxygen (O2) is absorbed from the air into the blood capillaries and the action of the heart circulates it through all the tissues in the body. At the same time, carbon dioxide (CO2) is transmitted from the blood capillaries into the alveoli and then expelled through the bronchi and the upper respiratory tract.
  \item {\bf Diaphragm} muscular partition between the chest and the abdominal cavity
  \item {\bf Process of Inhalation}
    \begin{itemize}
      \item The diaphragm contracts and moves down
      \item The rib muscles contract and cause the ribs to move outward
      \item The volume of the chest cavity increases
      \item Air pressure in the lungs decreases (Boyle’s Law)
      \item The difference in air pressure between the lungs and
        outside air causes air to rush into the lungs.
    \end{itemize}
  \item {\bf Process of Exhalation} Passive event.
    \begin{itemize}
    \item The diaphragm relaxes
    \item The size of the chest cavity decreases
    \item Pressure in the chest cavity is greater than atmospheric
      pressure.
    \item Air is pushed out of the lungs.
    \end{itemize}
  \item {\bf How breathing is controlled} Controlled by medulla
    oblongota (in brain). Looks at $CO_2$. If high, then makes
    diaphragm contract. Get air in. More $CO_2$, more impulse to
    breathe in.
  \item {\bf Tobacco and the respiratory system} Three dangerous
    things:
    \begin{itemize}
    \item Nicotine: Increases BP and heart rate. Stimulant.
    \item Carbon Monoxide: Sticks to hemoglobin. So oxygen cannot stick. So cells starved for oxygen.
    \item Nicotine and Carbon Monoxide: Paralyze Cilia
    \item Tar: Cause cancer (carcinogenic)
    \end{itemize}
\end{enumerate}
\section*{Chapter 38: (No 38-1)}
\begin{enumerate}
  \item {\bf Process of digestion from mouth to large intestine including accessory structures}
  \item {\bf Mechanical digestion} Chewing to break down large
    pieces. Teeth cut and tear into food.
  \item {\bf Chemical digestion}  Use enzymes to break down food.
  \item {\bf Amylase} Enzyme in saliva that breaks the chemical bonds
    in starches and releases sugars
  \item {\bf Lysozyme} Enzyme in saliva that fights infections by
    digesting bacterial cell walls.  in starches and releases sugars
  \item {\bf Pepsin} Enzyme that digests protein. Pepsin works best
    under acidic conditions. Released in stomach.
  \item {\bf Structures of excretory system}
  \item {\bf Filtration}
  \item {\bf Reabsorption}
  \item {\bf Urine composition} Urea, excess salt and water.
  \item {\bf Water balance}
\end{enumerate}
\section*{Chapter 39}
\begin{enumerate}
  \item {\bf Hormones} Coordinate slower but longer-acting responses
    including reproduction, development, energy metabolism, growth,
    and behavior. Secreted by endocrine system.
  \item {\bf Exocrine glands vs Endocrine glands}
    \begin{itemize}
      \item Exocrine glands release secretions through ducts directly
        to the organs that use them. Eg: glands that release sweat,
        tears, digestive juices – effect is localized.
      \item Endocrine glands release their secretions directly into
        the bloodstream – can affect cells throughout the body
    \end{itemize}
  \item {\bf Regulation of Thyroxine}
  \begin{enumerate}
    \item High Thyroxine - 1. no TRH 2. no TSH 3. no thyroxine
    \item Low Thryoxine - 1. Hypothalamus secretes TRH 2. TRH simulates pituitary to secrete TSH 3. TSH stimulates release of thyroxine
  \end{enumerate}
  \item {\bf Regulation of water}
  \begin{enumerate}
    \item Water Oversupply, less ADH released, kidneys increase water removal
    \item Water level low, hypothalamus signals pituitary, pituitary releases ADH, Blood carries ADH to kidneys, kidneys decrease removal of water, brains sends the signal of thirst
  \end{enumerate}
  \item {\bf Regulation of calcium}
  \begin {enumerate}
    \item Calcium High - thyroid secretes calcitonin, kidneys reabsorb less calcium, reduces amount of calcium absorbed
    \item Calcium Low - PTH is released, kidneys absorb more calcium from food
  \end{enumerate}
  \item {\bf Regulation of blood glucose}
  \begin{enumerate}
    \item High Glucose - pancreas secrete insulin, cells take glucose out of blood
    \item Low Glucose - Glucogen released from pancreas, stimulates liver and skeletal muscles to break down glycogen, causes fat cells to break down fats
  \end{enumerate}
  \item {\bf Structure of sperm} Head, mid-peice, tail
  \item {\bf Structures of the male reproductive system}
  \begin{enumerate}
    \item scrotum
    \item testes
    \item epididymis
    \item vas deferens
    \item Seminal vesicles
    \item Urethra
    \item Prostate gland
    \item Bulbourethral gland
  \end{enumerate}
  \item {\bf Structures of the female reproductive system}
  \begin{enumerate}
    \item Fallopian tube
    \item Ovary
    \item Uterus
    \item Urinary bladder
    \item pubic bone
    \item urethra
    \item cervix
    \item rectum
    \item vagina
  \end{enumerate}
  \item {\bf Menstrual cycle phases (including hormones)}
    \begin{itemize}
    \item follicular phase, ovulation, luteal phase, menstruation
    \item Horomones: FSH, Estrogen, LH, Progesterone
    \end{itemize}
  \item {\bf Fertilization (what this is and where it takes place)}
    The process of a sperm joining an egg. Sperm swims up the
    fallopian tues from vagina. If egg is in the fallopian tube,
    fertilization likely.
  \item {\bf Zygote}
  A fertilized egg.
  \item {\bf Function of placenta} Connects the mother and developing
    embryo. Embryo gets its oxygen and nutrients and excretes its
    waste products. Acts as a barrier to some harmful or
    disease-causing agents. But HIV, measles can cross. Also drugs and
    alcohol can penetrate placenta.
  \item {\bf Hormones involved in child birth and milk production}
    \begin{itemize}
      \item: Signal baby is ready. Synchronize uterine
        contractions. Dilate the cervix. Prepare mother for nursing.
      \item Prolactin stimulates production of milk. (Horomone
        secreted by pituatary glands).
      \item Oxytocin: Affects involuntary muscles in uterine
        walls. They contract rhythmically - called labor.
    \end{itemize}    
  \item {\bf Identical twins vs fraternal twins}
    \begin{itemize}
    \item Fraternal: If two eggs are released during the same cycle
      and fertilized by two different sperm
    \item Identical: A single zygote may split apart to produce two
      embryos
    \end{itemize}
\end{enumerate}
\section*{Chapter 40: ( No 40-1 and 40-4)}
\begin{enumerate}
  \item {\bf Non-specific defenses}
    Defenses are designed to keep most foreign things out of the body.
  \item {\bf First line of defense}
  \begin{itemize}
    \item Intact skin: Epidermis shield against invaders. Secrete
      chemicals that kill invaders. pH maintained between 3 and
      5. Acidic to prevent microbe colonization.
    \item Mucus and Cilia: Mucus is viscous fluid and traps bacteria
      and foreign particles. Hair like cilia sweeps this goop into
      throat for coughing and swallowing. Keep goop away from lungs.
    \item Saliva: Contain lysozyme to break bacterial cell wall (kill
      bacteria)
    \item Tears: Also has lysozyme
    \item Stomach Acids: Strong acids break down swallowed bacteria
  \end{itemize}
  \item {\bf Second line of defense}
    \begin{itemize}
      \item Blood WBC : Attack invaders. Initiate inflammatory
        response to protect tissue. Phagocytes: Engulf and destroy
        bacteria in lysosomes.
      \item Fever: High temp means pathogen growth slowed or
        stopped. High temp increases heart rate so WBC can get to
        infection place faster.
      \item Interferon: When virus infection body releases
        interferon. Interferes with virus replication. Block synthesis
        of key proteins required for viral replication.
    \end{itemize}
  \item {\bf Antigens} Any substance, as a virus or bacterium, that
    triggers immune response. Specific Defense.
  \item {\bf Antibodies} Protein that recognizes and binds to an
    antigen.
  \item {\bf Risks of transplant} Immune system thinks transplant in
    foreign and attacks it. Makes transplants tough. Cells have marker
    proteins on surface which lets immune system recognize them. Can
    cause organ rejection.
  \item {\bf Passive Immunity} Antobodies produced by other animals
    are injected into the bloodstream. Eg. Mom passing down immunity
    to baby via breast milk or placenta.
  \item {\bf Active Immunity} You produce the antibodies - Your body
    has been exposed to the antigen in the past either by infection or
    vaccination.
  \item {\bf Allergies} Immune system mistakenly recognizes harmless
    foreign particles as serious threats.
  \item {\bf What is histamine?}  Allergens attach to mast cells. Mast
    cells in locations where external contact possible. Mast cells
    start inflammatory response. These release histamines. Histamines
    increase blood flow and fluid production in affected
    area. Sneezing, watery eyes, runny nose.
  \item {\bf Humoral immunity} Defends the body against antigens and
    pathogens in body fluids (blood and lymph) - involves mainly B
    cells. Produce antibody.
  \item {\bf Cell mediated immunity} Defends the body against abnormal
    cells and pathogens inside living cell - involves mainly T
    cells. No antibody production.
  \item {\bf Secondary immune response} When $2^{nd}$ time person
    exposed to antigen, immunological memory is triggered and immune
    system can start making antibodies immediately.
  \item {\bf Autoimmune disorder} When the immune system attacks the
    body's own cells, it produces an autoimmune disease. Thinks body's
    own cells are pathogens.
  \item {\bf Immunodeficiency disorder} Person has a weakened immune
    response. E.g. AIDs caused by HIV.
\end{enumerate}
\end{document}
